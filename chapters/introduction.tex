\thispagestyle{plain}
\newgeometry{left=4cm,bottom=2.5cm}
\section{\raggedleft Introduction générale}
\vspace{1cm}
\setlength{\parindent}{40pt}
\justifying
\begin{small}
    \lettrine[findent=2pt,lines=3]{\textit{L}}{}e marché mondial de location de voitures continue, chaque année, à se développer et à s'amplifier, et le besoin d'atteindre un nombre maximal de clients est devenu une nécessité. Surtout avec l'avancement technologique qui a permis a tous le monde d'avoir un smarthphone. Cet appareil capable de réaliser plusieurs fonctionnalités en simples clics.

    \noindent Pendant les années précédentes, avec la pandémie mondiale de COVID-19, qui a forcé des milliards de personnes à rester chez eux, il est devenu indispensable de digitaliser plusieurs services afin de les rendre plus accessible, plus fiable, et continuer à exister dans un marché qui continue à évoluer et innover. Plusieurs personnes on pû se bénéficier de ses services sans quitter leurs maisons et risquer être atteints par la COVID, grâce aux différentes entreprises qui ont choisi de continuer à servir leurs clients à l'aide des applications mobiles.

    \noindent L'entreprise Swiss Premium Negoce a vu cet évênement mondial comme une opportunité pour s'introduire dans le monde vaste de nouvelles technologies et présenter leurs différents servies sous forme numérique à l'aide des sites web et applications mobiles. Cette étape qui les permettra de rester toujours proche de sa clientèle, et continuer à fonctionner sans tenir compte de la pandémie.

    \noindent L'un des services offerts par Swiss Premium Negoce est le service de location de voitures de luxe. Ce service permet aux utilisateurs de trouver leurs voitures de rêves et la louer même pour une courte durée avec la possibilité d'avoir un chauffeur personnel. tout cela sera possible à l'aide de l'application Swiss Premium Negoce : Cars ou tout simplement : SPN Cars. Le présent rapport qui présentera cette application, s'étalera sur quatre chapitres:

    \begin{itemize}
        \item Le premier chapitre présentera l'organisme d'accueil, ainsi le projet à réaliser et une étude sur les différents aspects de ce projet.
        \item Le deuxième chapitre fera l'objet de présenter les différents besoins qui seront achevés par cette application.
        \item Le troisième chapitre sera à propos la conception de l'application, et les technologies choisies pour réaliser le projet suite à cette application.
        \item Le quatrième et dernier chapitre présentera la réalisation de ce projet, qui est une explication détaillée de chacune des fonctionnalités offerts par l'application avec des captures d'écran pour illustrer le travail réalisé.
    \end{itemize}

    \noindent Le projet sera clôturé par une conclusion générale.
\end{small}