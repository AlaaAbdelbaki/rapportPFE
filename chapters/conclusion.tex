\newgeometry{left=4cm,bottom=2.5cm}
\chapter*{Conclusion et perspecives}
\markboth{Conclusion et perspecives}{}
\addcontentsline{toc}{chapter}{Conclusion et perspectives}
\vspace{1cm}
\justifying
\lettrine[findent=2pt,lines=3]{\textit{D}}{}ans le présent rapport, on a dressé le bilan complet de ce travail qui se situe dans le cadre de mon projet de fin d'études. L'objectif principal de ce projet étant de concevoir et de développer une application mobile de location de voitures de luxes pour les clients de Swiss Premium Negoce, l'entreprise acceuillante.

On a entamé ce rapport par une présentation du cadre général du projet où on a présenté l'entreprise acceuillante et ses activitées, puis on a présenté des différentes applications qui offrent des services de location de voitures et service taxi, ses applications on présenté des inconvénients. On a dégagé d'après ses inconvénients la problématique ayant engendré le besoin d'unne telle application. Pour réaliser cette application, on a dû ensuite séparer ses boisoins en besoins fonctionnels et non fonctionnels qui ont permis de commencer l'étape de conception, où on a décrit les différentes fonctionnalités nécessaires. Cette étape a permis d'identifier les technologies et outils à utiliser pour arriver à réaliser ce projet dans les délais choisis.\\
\noindent Une fois les outils et technologies choisies, on a passé au développement de l'application, où on a commencé par le développement des interfaces graphiques avec Flutter et leurs fonctionnalités avec Javascript et ExpressJS en coté serveur, au fin de développement de chaque interface et ses fonctionnalités, elle est testée pour dégager les différentes erreurs qui peuvent se déclancher et les corriger, jusqu'au développement de la dernière interface de l'application.

Ce projet fut une expérience enrichissante et fructueuse qui m'a permis d'acquérir de nouvelles compétences de grande valeur dans plusieurs domaines à la fois : Le développement web, davelopmmenet mobile et le design des interfaces graphique et l'expérience utilisateur (UI/UX Design), en particulier le développement des applications coté serveur avec ExpressJS, développement des applications mobiles avec Flutter et la création des maquettes des interfaces utilisateur avec Adobe XD. Ainsi, ce stage qui représente une nouvelle expérience professionnelle, m'a été bénéfique, il a été une nouvelle chance pour consolider mes compétences techniques et mettre en pratique mes compétences théoriques.

J'ai appris à être autonome, ouvert aux autres, et plus conscient de la vie professionnelle qui nécessite la ponctualité, le sérieux et le travail acharné. J'ai pris la responsabilité et j'ai développé l'application qui répondait aux besoins définis dans le cahier de charges, les exigences, et les délais.

Comme tout projet informatique, des contraints on été rencontrées et elles ont été surmontées tout au long de la réalisation du projet. Elles ont constitué un défi pour acquérir de nouvekkes connaissances techniques et de développer les capacités opérationnelles. Ces défis concernent notamment la familiarisation avec les nouveaux frameworks et la maîtrise de certains outils techniques.

Pour le futur de cette application, on a plusieurs objectifs planifiés, dont one cite :
\begin{itemize}
    \item Une migration vers firebase pour utiliser tous les services offerts par cette plateforme réduire le nombre de bases de données utilisées.
    \item Le développement des applications SPN-Cars Driver et SPN-Cars Admin où il y aure une communication entre ces trois applications et assurer une service rapide et de qualité pour le client.
\end{itemize}
Ses objectifs, une fois atteints, permettront d'améliorer l'expérience de l'utilisateur et vont ouvrir les portes pour trouver plusieurs autres améliorations.

J'espère que le présent rapport soit suffisamment clair et structuré pour sue le lecteur ait une idée précise et complète sur les différentes tâches que j'ai effectuées.