\chapter{Spécification des besoins}
\minitoc
\clearpage
\section*{Introfuction}
L'application \textbf{SPN-Cars} est une application mobile permettant à ses utilisateurs de louer des voitures luxueuses avec ou sans chauffeur pour différents types de voyages : Transfert - Excursion - ou tout simplement louer le véhicule.\\
\noindent Avant de passer au développement de l'application, il est nécessaire de passer par l'étape de conception où il faut déterminer les acteurs et leurs besoins : fonctionnels et non fonctionnels.
\section{Spécification des besoins fonctionnels}
\begin{itemize}
    \item L'authentification pour accéder aux différents services offerts par l'application.
    \item Consulter les voitures disponibles selon la position actuelle de l'utilisateur.
    \item Paiement en ligne.
    \item Louer plusieurs voitures à la fois.
    \item Consulter la position des voitures louées en temps réel.
    \item Contacter les chauffeurs par messagerie instantanée.
\end{itemize}
\section{Spécification des besoins non fonctionnels}
\begin{itemize}
    \item Interfaces d'utilisateurs agréables et simple à utiliser.
    \item Interactions fluide avec l'ensemble des services de l'application.
    \item L'application doit être facile à maintenir dans le futur.
\end{itemize}
